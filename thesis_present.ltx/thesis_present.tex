\documentclass{beamer}

\setbeamertemplate{navigation symbols}{}

\usepackage[utf8]{inputenc}
\usepackage[english,greek]{babel}
\usepackage{array}

\date{\today}
\author{Δημήτρης Σαριδάκης Μπίτος}
\institute{Εθνικό Μετσόβιο Πολυτεχνείο}

\def\e{\foreignlanguage{english}}
\def\h{\e{Haskell}}
%\newcommand\_[_]{_}
\newcolumntype{P}[1]{>{\centering\arraybackslash}p{#1}}

\begin{document}

\title{Σχεδίαση και Υλοποίηση της Γλώσσας Προγραμματισμού \e{lambda-cases}}
\frame{\titlepage}

\begin{frame}

\frametitle{Μ' αρέσει πολύ η \h}

Τα πάντα είναι τιμές (όροι) και έχουν κάποιο τύπο:

\begin{itemize}

\item Σταθερές
\item Συναρτήσεις
\item Είσοδος/Έξοδος

% \pause
\begin{itemize}

\item Άρα μπορούν να είναι ορίσματα συναρτήσεων, στοιχεία λίστας κτλ
\\~\
\end{itemize}

\end{itemize}

% \pause
Οι τύποι τα λένε όλα.
\\~\

% \pause
Βοηθητικός μεταγλωττιστής:

\begin{itemize}

\item Μεταγλωττίζεται? Δουλεύει! (Συνήθως)

\item Δεν μεταγλωττίζεται? Οι τάδε τύποι δεν ταιριάζουν.
\\~\

\end{itemize}

% \pause
Γιατί να γράψω κώδικα σε άλλη γλώσσα?

\end{frame}

\begin{frame}

\frametitle{Γιατί δεν είναι η πιο διαδεδομένη γλώσσα?}

\e{Building} εργαλεία όχι τόσο καλά:

\begin{itemize}

\item Φαίνεται να υπάρχει βελτίωση (απ'όσο λένε \e{online})

\item Δεν αφορά την διπλωματική
\\~\

\end{itemize}

% \pause

Δύσκολη στην εκμάθηση για αρχάριο. Ίσως παίζουν ρόλο:

\begin{itemize}

\item Όχι πολύ περιγραφικές λέξεις κλειδιά

\item Όχι πολύ περιγραφικά ονόματα βασικών συναρτήσεων

\item Γραμματική λάμδα λογισμού

\end{itemize}

\end{frame}

\begin{frame}

\frametitle{Τι θα άλλαζα για μένα?}

Μπορούν να συμπτυχθούν κομμάτια που γράφω πολύ συχνά?
\begin{itemize}

\item Ορισμοί Τιμών

\item \e{LambdaCase extension}
\\~\

\end{itemize}

Μπορούν να αλλάξουν κομμάτια ώστε να μοιάζουν περισσότερο στα αντίστοιχα άλλων
γλωσσών όπου είναι πιο κατανοητά?
\begin{itemize}

\item Τελεία για \e{attributes/members/fields}

\item Εφαρμογή συνάρτησης με ορίσματα σε παρένθεση
\\~\

\end{itemize}

Υπάρχει κάτι καινούργιο που θα μπορούσα να προσθέσω?
\begin{itemize}

\item Ορίσματα στην αρχή ή στην μέση του ονόματος της συνάρτησης

\item Ανώνυμες Παράμετροι
\\~\

\end{itemize}
Μπορώ να δώσω πιο κατανοητά ονόματα?

\end{frame}

\begin{frame}[fragile]

\frametitle{Εφαρμογή Συνάρτησης Με Παρενθέσεις}

\begin{otherlanguage}{english}

\begin{center}
\begin{tabular}{ |P{5cm}|P{5cm}| }
 \hline
 Haskell & lcases
 \\
 \hline
 \verb|f x| & \verb|f(x)|
 \\
 \verb|g x y z| & \verb|g(x, y, z)|
 \\
 \verb|putStrLn "Hello World!"| & \verb|print("Hello World!")|
 \\
 \hline
\end{tabular}
\end{center}

% \pause
\begin{center}
\begin{tabular}{ |P{5cm}|P{5cm}| }
 \hline
 \verb|show x| & \verb|(x)to_string|
 \\
 \verb|mod x y| & \verb|(x)mod(y)|
 \\
 \verb|map f l| & \verb|apply(f)to_all_in(l)|
 \\
 \hline
\end{tabular}
\end{center}

\end{otherlanguage}
Ορισμό \e{apply to all}?

\end{frame}

\begin{frame}[fragile]

\frametitle{Ανώνυμες Παράμετροι}


\begin{otherlanguage}{english}

\verb|f(x, y, _)|, \verb|f x y|
\\~\

\verb|f(x, _, z)|, \verb|\y -> f x y z|
\\~\

\verb|f(_, y, z)|, \verb|\x -> f x y z|
\\~\

\verb|f(_, _, z)|, \verb|\x y -> f x y z|

\end{otherlanguage}


Παραδείγματα
\\~\

και για λίστες \e{tuples}

\end{frame}

\begin{frame}[fragile]

\frametitle{Ορισμοί \e{tuple\_type} και \e{postfix functions}}

\e{tuple\_type} αντίστοιχα:

\begin{itemize}

\item \e{structs} σε \e{C}

\item \e{classes} σε \e{OOP}: μόνο \e{attributes}

\item \e{records} σε \h
\\~\

\end{itemize}

Δημιουργείται αυτόματα ένα \e{postfix function} για κάθε \e{field}:

\begin{itemize}

\item Κατευθείαν με όρισμα:
\begin{otherlanguage}{english}
\verb|some_person.last_name|
\end{otherlanguage}

\item Συνάρτηση Μόνη της:
\begin{otherlanguage}{english}
\verb|_.last_name|
\end{otherlanguage}

\end{itemize}

\end{frame}

\begin{frame}[fragile]

\frametitle{Ορισμοί \e{tuple\_type} και \e{postfix functions}}

\begin{otherlanguage}{english}
\begin{verbatim}
tuple_type Name
value (first_name, last_name) : String^2
\end{verbatim}
% \pause
\begin{verbatim}
awesome_guy: Name
  = ("Leonhard", "Euler")
\end{verbatim}
% \pause
\begin{verbatim}
>>> awesome_guy.last_name
  : String
  = "Euler"
\end{verbatim}
% \pause
\begin{verbatim}
>>> _.last_name
  : Name => String
\end{verbatim}
% \pause
\begin{verbatim}
>>> apply(_.last_name)to_all_in(_)
  : ListOf(Name)s => ListOf(String)s
\end{verbatim}
\end{otherlanguage}

\end{frame}

\begin{frame}

\frametitle{\e{postfix functions} για \e{product type tuples}}

Λίστα
\\~\

Παραδείγματα

\end{frame}

\begin{frame}

\frametitle{\e{".change" postfix function}}

Συναρτήση αλλαγής στοιχείων
\\~\

Παραδείγματα
\end{frame}

\begin{frame}

\frametitle{Ορισμοί \e{or\_type} και \e{prefix functions}}

Παραδείγματα
\\~\

\end{frame}

\begin{frame}

\frametitle{Τελεστές}

Πίνακας

\end{frame}

\begin{frame}

\frametitle{Εκφράσεις Συναρτήσεων}

\end{frame}

\begin{frame}

\frametitle{Εκφράσεις Συναρτήσεων \e{"cases"}}

\e{pattern matching}
\\~\

\e{LambdaCase extension}

\end{frame}

\begin{frame}

\frametitle{Ορισμοί Τιμών}

Σύγκριση με \h

\end{frame}

\begin{frame}

\frametitle{Εκφράσεις \e{"where"}}
Παραδείγματα

\end{frame}

\begin{frame}

\frametitle{Τύποι}

Αντιστοιχία με \h

\end{frame}

\begin{frame}

\frametitle{Πατσούκλια Τύπων}

\e{type} στην \h
\\~\

Παραδείγματα

\end{frame}

\begin{frame}

\frametitle{Λογική Τύπων}

Μηχανισμός \e{ad hoc} πολυμορφισμού στην \e{lcases}.
\\~\

Αντιστοιχοί στα \e{type classes}.

\end{frame}

\begin{frame}

\frametitle{Ορισμοί Προτάσεων Τύπων}

Ατομικές, \e{class}
\\~\

Μετονομασίας

\end{frame}

\begin{frame}

\frametitle{Θεωρήματα Τύπων}

\e{instance}

\end{frame}

\begin{frame}

\frametitle{Υλοποίηση \e{Parser}}

Βιβλιοθήκη \e{Parsec}

\end{frame}

\begin{frame}

\frametitle{Μετάφραση σε \h}

\end{frame}

\begin{frame}

\frametitle{Συμπεράσματα}

Τι έχει γίνει
\\~\

Τι θα ήταν καλό να γίνει

\end{frame}

\end{document}
